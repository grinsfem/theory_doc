\subsection{Background Information}
The derivation detailed herein is for computing analytical Jacobians for the Low Mach Navier-Stokes equations. The final results are then implemented in GRINS src/physics/src/low\_mach\_navier\_stokes.C, with the goal being that there
will be a significant speedup in computation over using the standard finite difference approximation. The derivation is broken up into 3 subsections: mass (P residual), momentum (U residual), and energy (T residual).
Note that a sign convention has been specified for purposes of optimization.\\
Also note that, in this case, the following parameters are considered functions of temperature, T:
\begin{itemize}
    \item density, $\rho$
    \item specific heat, $c_p$
    \item viscosity, $\mu$
    \item thermal conductivity, $k$
\end{itemize}
For the formulations, \textbf{u}=\{$P,U,T$\} and the test functions \textbf{$\Phi$}=\{$\xi,\phi,\psi$\}.\\
Lastly, the variable names used in the code follow this template: $Kxy$, where $x$ is the residual and $y$ is the derivation variable.  For example, $KPT$ is the derivative of the P residual with respect to temperature,
while $KuP$ is the derivative of the $u$-velocity component of the U residual with respect to pressure. 

\newpage
%%%%%%%%%%%%%%%%%%%%%%%%%%%%%%%%%%%%%%%%%%%%%%%%
\subsection{Mass Equation}
The given P residual:
\begin{equation}
    F_{p,i} = (-(U \cdot \nabla T) T^{-1} + \nabla \cdot U)\xi_i
\end{equation}

Now converting to tensor notation

\begin{equation}
    F_{p,i} = -u_a T_{a,b} T^{-1} \xi_i + u_{a,b}\xi_i
\end{equation}

Galerkin approximation:
\begin{align*}
    u_a &= \sum_i u_i \phi_i^a\\
    T &= \sum_i T_i \psi_i
\end{align*}

\begin{equation}
    F_{p,i} = -u_k \phi_k^a T_l \psi_l^{a,b} (T_m \psi_m)^{-1} \xi_i + u_n \phi_n^{a,b} \xi_i
\end{equation}

The P residual is a function of velocity, $U$, and temperature, $T$, so derivatives must be taken with respect to both of them using the following form:

\begin{align*}
    \frac{\partial}{\partial u_j} (u_i \phi_i^a) &= \phi_j^a \\
    \frac{\partial}{\partial T_j} (T_i \psi_i) &= \psi_j
\end{align*}

\begin{align}
    \frac{\partial F_{p,i}}{\partial u_j} = &-\phi_j^{a} T_l \psi_l^{a,b} (T_m \psi_m)^{-1} \xi_i + \phi_j^{a,b} \xi_i \nonumber \\
    \frac{\partial F_{p,i}}{\partial u_j} = &-\phi_j^{a} T_{a,b} T^{-1} \xi_i + \phi_j^{a,b} \xi_i \label {fp_du}\\ 
    \nonumber \\
    \frac{\partial F_{p,i}}{\partial T_j} = &-u_k \phi_k^a \xi_i \left [ T_l \psi_l^{a,b} (-1) (T_m \psi_m)^{-2} (\psi_j) + (T_m \psi_m)^{-1} \psi_j^{a,b} \right ] \nonumber \\
    \frac{\partial F_{p,i}}{\partial T_j} = &-u_a \xi_i \left [ -T_{a,b} T^{-2} \psi_j + T^{-1} \psi_j^{a,b} \right ] \label{fp_dT}
\end{align}

Equations \ref{fp_du} and \ref{fp_dT} above are the generalized Jacobian terms for the P residual. When implemented in LMNS.C, the following terms will result:\\
$KPu, KPv, KPw, KPT$

\newpage
%%%%%%%%%%%%%%%%%%%%%%%%%%%%%%%%%%%%%%%%%%%%%%%%%%%%%%
\subsection{Momentum Equation}
The U residual is given in tensor notation as follows:
\begin{equation}
    F_U = -\rho u_a u_{a,b} \phi^b + P \phi^{a,a} - \mu \left [u_{a,b} \phi_{a,b} + u_{b,a} \phi_{a,b} - \frac{2}{3} u_{a,a} \phi_{b,b} \right ] + \rho g_a \phi_a
\end{equation}

Galerkin approximation

\begin{align*}
    u_a &= \sum_i u_i \phi_i^a\\
    P &= \sum_i P_i \xi_i\\
    T &= \sum_i T_i \psi_i
\end{align*}

\begin{align}
    F_U =  -\rho u_i \phi_i^a u_j \phi_j^{a,b} \phi^b + P_i \xi_i \phi^{a,a} - \mu \left [u_i \phi_i^{a,b} \phi_{a,b} + u_i \phi_i^{b,a} \phi_{a,b} - \frac{2}{3} u_i \phi_i^{a,a} \phi_{b,b} \right ] + \rho g_a \phi_a \label{u_resid}
\end{align}

Equation \ref{u_resid} above is a function of velocity, pressure, and temperature (due to $\rho = \rho(T)$ \\and $\mu = \mu(T)$ \ ), so derivatives must be taken with respect to all of them.

\begin{align}
   \frac{\partial F_U}{\partial u_m} = &-\rho \left [ u_i \phi_i^a \phi_m^{a,b} + u_j \phi_j^{a,b} \phi_m^a \right ] \phi^b 
        - \mu \left [\phi_m^{a,b} \phi^{a,b} + \phi_m^{b,a} \phi^{a,b} - \frac{2}{3} \phi_m^{a,a} \phi^{b,b} \right ] \nonumber \\     
   \frac{\partial F_U}{\partial u_m} = &-\rho \left [ u_a \phi_m^{a,b} + u_{a,b} \phi_m^a \right ] \phi^b - \mu \left [\phi_m^{a,b} \phi^{a,b} + \phi_m^{b,a} \phi^{a,b} - \frac{2}{3} \phi_m^{a,a} \phi^{b,b} \right ] \label{du_du}\\
   \nonumber \\ 
   \frac{\partial F_U}{\partial P_m} = &\xi_m \phi^{a,a} \label{du_dp} \\
   \nonumber \\
   \frac{\partial F_U}{\partial T_m} = &-\frac{d\rho}{dT} \frac{d}{dT_m}(T_i \psi_i) u_i \phi_i^a u_j \phi_j^{a,b} \phi^b
        - \left[ 0 + \frac{d\mu}{dT} \frac{d}{dT_m}(T_i \psi_i) [u_i \phi_i^{a,b} \phi_{a,b} + u_i \phi_i^{b,a} \phi_{a,b} - \frac{2}{3} u_i \phi_i^{a,a} \phi_{b,b} ] \right ] \nonumber
        \\ &+ \frac{d\rho}{dT} \frac{d}{dT_m}(T_i \psi_i) g_a \phi_a \nonumber \\       
   \frac{\partial F_U}{\partial T_m} = &-\frac{d\rho}{dT} \psi_m u_a u_{a,b} \phi^b
        - \frac{d\mu}{dT} \psi_m \left [u_{a,b} \phi_{a,b} + u_{b,a} \phi_{a,b} - \frac{2}{3} u_{a,a} \phi_{b,b} \right ]
        + \frac{d\rho}{dT} \psi_m g_a \phi_a \label{du_dT} 
\end{align}

Equations \ref{du_du}, \ref{du_dp}, and \ref{du_dT} above can then be implement in LMNS.C to create the following terms:\\
$Kuu, Kuv, Kuw, Kvu, Kvv, Kvw, Kwu, Kwv, Kww, KuT, KvT, KwT, KuP, KvP, KwP$

\newpage
%%%%%%%%%%%%%%%%%%%%%%%%%%%%%%%%%%%%%%%%%%%%%%%%%%%%%
\subsection{Energy Equation}
The T residual is given as follows:
\begin{equation}
    F_T = -\rho c_p (u_a T_{a,b})\psi - k T_{a,b} \psi_{a,b}
\end{equation}

Again, note that $\rho = \rho(T)$, $c_p = c_p(T)$, and $k = k(T)$

\begin{align}
    F_T = -\rho c_p u_i \phi_i^a T_j \psi_j^{a,b} \psi_k - k T_i \psi_i^{a,b} \psi_j^{a,b}
\end{align}

Noting $F_T$ is a function of velocity and temperature.

\begin{align}
    \frac{\partial F_T}{\partial u_m} = &-\rho c_p \phi_m^a T_j \psi_j^{a,b} \psi_k - 0 \nonumber \\
    \frac{\partial F_T}{\partial u_m} = &-\rho c_p \phi_m^a T_{a,b} \psi_k \label{dT_du} \\
    \nonumber \\
    \frac{\partial F_T}{\partial T_m} = &-u_i \phi_i^a \psi_k \left [ \rho [ c_p \psi_m^{a,b} + T_j \psi_j^{a,b} \frac{d c_p}{dT} \psi_m ] + [ c_p T_j \psi_j^{a,b} ] \frac{d \rho}{dT} \psi_m  \right ] \nonumber \\
                                      &- k \psi_m^{a,b} \psi_j^{a,b} - T_i \psi_i^{a,b} \psi_j^{a,b} \frac{dk}{dT} \psi_m \nonumber \\
    \frac{\partial F_T}{\partial T_m} = &-u_a \psi_k \left [ \rho [ c_p \psi_m^{a,b} + T_{a,b} \frac{d c_p}{dT} \psi_m ] + [ c_p T_{a,b} ] \frac{d \rho}{dT} \psi_m  \right ] \nonumber \\
                                      &- k \psi_m^{a,b} \psi_j^{a,b} - T_{a,b} \psi_j^{a,b} \frac{dk}{dT} \psi_m \label{dT_dT}
\end{align}

Equations \ref{dT_du} and \ref{dT_dT} above will result in the following terms for LMNS.C:\\
$KTu, KTv, KTw, KTT$

\end{document}
