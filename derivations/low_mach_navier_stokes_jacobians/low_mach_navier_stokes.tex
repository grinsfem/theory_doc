\documentclass[10pt]{article}
 
\title{Derivation of Low Mach Navier-Stokes Jacobians}
\date{}

\usepackage{helvet}
\renewcommand*\familydefault{\sfdefault}
\usepackage[left=1in,right=1in,top=1.0in,bottom=1in]{geometry}
\usepackage[small,compact]{titlesec}
\usepackage{amsmath}
\usepackage{graphicx}

\begin{document}

\begin{center}
\textbf{Derivation of Low Mach Navier-Stokes Jacobians}
\end{center}

\section{Background Information}
The derivation detailed herein is for computing analytical Jacobians for the Low Mach Navier-Stokes equations. The final results are then implemented in GRINS src/physics/src/low\_mach\_navier\_stokes.C, with the goal being that there
will be a significant speedup in computation over using the standard finite difference approximation for computing the Jaconians. The derivation is broken up into 3 sections: mass (P residual), momentum (U residual), and energy (T residual).
Each of these is computed in a separate function wihtin the LMNS.C file.  The 3 residuals were already implemented, so they will be the starting point for this derivation. Note that a sign convention has been specified for 
purposes of optimization.\\
Also note that, in this case, the following parameters are considered functions of temperature, T:
\begin{itemize}
    \item density, $\rho$
    \item specific heat, $c_p$
    \item viscosity, $\mu$
    \item thermal conductivity, $k$
\end{itemize}
For the test functions, \textbf{u}=\{$P,U,T$\} and \textbf{$\Phi$}=\{$\xi,\phi,\psi$\}.\\
Lastly, the variable names used in the code follow this template: $Kxy$, where $x$ is the residual and $y$ is the derivation variable.  For example, $KPT$ is the derivative of the P residual with respect to temperature,
while $KuP$ is the derivative of the $u$-velocity component of the U residual with respect to pressue. 


%%%%%%%%%%%%%%%%%%%%%%%%%%%%%%%%%%%%%%%%%%%%%%%%
\section{Mass Equation}
The given P residual:
\begin{equation}
    F_p = (-(U \cdot \nabla T) T^{-1} + \nabla \cdot U)\xi
\end{equation}

Now converting to tensor notation

\begin{equation}
    F_p = -u_a T_{a,b} T^{-1} \xi + u_{a,b}\xi
\end{equation}

Galerkin approximation:
\begin{align*}
    u_a &= \sum u_i \phi_i^a\\
    T &= \sum T_i \psi_i
\end{align*}

\begin{equation}
    F_p = -u_i \phi_i^a T_j \psi_j^{a,b} (T_k \psi_k)^{-1} \xi + u_i \phi_i^{a,b} \xi
\end{equation}

The P residual is a function of velocity, $U$, and Temperature, $T$, so we must take derivative with respect to both of them using the following form

\begin{align*}
    \frac{\partial}{\partial u_m} (u_i \phi_i^a) &= \phi_m^a \\
    \frac{\partial}{\partial T_m} (T_i \psi_i) &= \psi_m
\end{align*}

\begin{align}
    \frac{\partial F_p}{\partial u_m} = &-\phi_m^{a} T_j \psi_j^{a,b} (T_k \psi_k)^{-1} \xi + \phi_m^{a,b} \xi \nonumber \\
    \frac{\partial F_p}{\partial u_m} = &-\phi_m^{a} T_{a,b} T^{-1} \xi + \phi_m^{a,b} \xi \label {fp_du}\\ 
    \nonumber \\
    \frac{\partial F_p}{\partial T_m} = &-u_i \phi_i^a \xi \left [ T_j \psi_j^{a,b} (-1) (T_k \psi_k)^{-2} (\psi_m) + (T_k \psi_k)^{-1} \psi_m^{a,b} \right ] \nonumber \\
    \frac{\partial F_p}{\partial T_m} = &-u_a \xi \left [ -T_{a,b} T^{-2} \psi_m + T^{-1} \psi_m^{a,b} \right ] \label{fp_dT}
\end{align}

Equations \ref{fp_du} and \ref{fp_dT} above are the generalized Jacobian terms for the P residual. When implemented in LMNS.C, the following terms will result:\\
$KPu, KPv, KPw, KPT$


%%%%%%%%%%%%%%%%%%%%%%%%%%%%%%%%%%%%%%%%%%%%%%%%%%%%%%
\section{Momentum Equation}
The U residual is given as follows:
\begin{equation}
    F_U = -\rho u_a u_{a,b} \phi^b + P \phi^{a,a} - \mu \left [u_{a,b} \phi_{a,b} + u_{b,a} \phi_{a,b} - \frac{2}{3} u_{a,a} \phi_{b,b} \right ] + \rho g_a \phi_a
\end{equation}

Galerkin approximation

\begin{align*}
    u_a &= \sum u_i \phi_i^a\\
    P &= \sum P_i \xi_i\\
    T &= \sum T_i \psi_i
\end{align*}

\begin{align}
    F_U =  -\rho u_i \phi_i^a u_j \phi_j^{a,b} \phi^b + P_i \xi_i \phi^{a,a} - \mu \left [u_i \phi_i^{a,b} \phi_{a,b} + u_i \phi_i^{b,a} \phi_{a,b} - \frac{2}{3} u_i \phi_i^{a,a} \phi_{b,b} \right ] + \rho g_a \phi_a \label{u_resid}
\end{align}

Equation \ref{u_resid} above is a function of velocity, pressure, and temperature (due to $\rho = \rho(T)$ \\and $\mu = \mu(T)$ \ ), so derivatives must be taken with respect to all of them.

\begin{align}
   \frac{\partial F_U}{\partial u_m} = &-\rho \left [ u_i \phi_i^a \phi_m^{a,b} + u_j \phi_j^{a,b} \phi_m^a \right ] \phi^b 
        - \mu \left [\phi_m^{a,b} \phi^{a,b} + \phi_m^{b,a} \phi^{a,b} - \frac{2}{3} \phi_m^{a,a} \phi^{b,b} \right ] \nonumber \\     
   \frac{\partial F_U}{\partial u_m} = &-\rho \left [ u_a \phi_m^{a,b} + u_{a,b} \phi_m^a \right ] \phi^b - \mu \left [\phi_m^{a,b} \phi^{a,b} + \phi_m^{b,a} \phi^{a,b} - \frac{2}{3} \phi_m^{a,a} \phi^{b,b} \right ] \label{du_du}\\
   \nonumber \\ 
   \frac{\partial F_U}{\partial P_m} = &\xi_m \phi^{a,a} \label{du_dp} \\
   \nonumber \\
   \frac{\partial F_U}{\partial T_m} = &-\frac{d\rho}{dT} \frac{d}{dT_m}(T_i \psi_i) u_i \phi_i^a u_j \phi_j^{a,b} \phi^b
        - \left[ 0 + \frac{d\mu}{dT} \frac{d}{dT_m}(T_i \psi_i) [u_i \phi_i^{a,b} \phi_{a,b} + u_i \phi_i^{b,a} \phi_{a,b} - \frac{2}{3} u_i \phi_i^{a,a} \phi_{b,b} ] \right ] \nonumber
        \\ &+ \frac{d\rho}{dT} \frac{d}{dT_m}(T_i \psi_i) g_a \phi_a \nonumber \\       
   \frac{\partial F_U}{\partial T_m} = &-\frac{d\rho}{dT} \psi_m u_a u_{a,b} \phi^b
        - \frac{d\mu}{dT} \psi_m \left [u_{a,b} \phi_{a,b} + u_{b,a} \phi_{a,b} - \frac{2}{3} u_{a,a} \phi_{b,b} \right ]
        + \frac{d\rho}{dT} \psi_m g_a \phi_a \label{du_dT} 
\end{align}

Equations \ref{du_du}, \ref{du_dp}, and \ref{du_dT} above can then be implement in LMNS.C to create the following terms:\\
$Kuu, Kuv, Kuw, Kvu, Kvv, Kvw, Kwu, Kwv, Kww, KuT, KvT, KwT, KuP, KvP, KwP$


%%%%%%%%%%%%%%%%%%%%%%%%%%%%%%%%%%%%%%%%%%%%%%%%%%%%%
\section{Energy Equation}
The T residual is given as follows:
\begin{equation}
    F_T = -\rho c_p (u_a T_{a,b})\psi - k T_{a,b} \psi_{a,b}
\end{equation}

Again, note that $\rho = \rho(T)$, $c_p = c_p(T)$, and $k = k(T)$

\begin{align}
    F_T = -\rho c_p u_i \phi_i^a T_j \psi_j^{a,b} \psi_k - k T_i \psi_i^{a,b} \psi_j^{a,b}
\end{align}

Noting $F_T$ is a function of velocity and temperature.

\begin{align}
    \frac{\partial F_T}{\partial u_m} = &-\rho c_p \phi_m^a T_j \psi_j^{a,b} \psi_k - 0 \nonumber \\
    \frac{\partial F_T}{\partial u_m} = &-\rho c_p \phi_m^a T_{a,b} \psi_k \label{dT_du} \\
    \nonumber \\
    \frac{\partial F_T}{\partial T_m} = &-u_i \phi_i^a \psi_k \left [ \rho [ c_p \psi_m^{a,b} + T_j \psi_j^{a,b} \frac{d c_p}{dT} \psi_m ] + [ c_p T_j \psi_j^{a,b} ] \frac{d \rho}{dT} \psi_m  \right ] \nonumber \\
                                      &- k \psi_m^{a,b} \psi_j^{a,b} - T_i \psi_i^{a,b} \psi_j^{a,b} \frac{dk}{dT} \psi_m \\
    \frac{\partial F_T}{\partial T_m} = &-u_a \psi_k \left [ \rho [ c_p \psi_m^{a,b} + T_{a,b} \frac{d c_p}{dT} \psi_m ] + [ c_p T_{a,b} ] \frac{d \rho}{dT} \psi_m  \right ] \nonumber \\
                                      &- k \psi_m^{a,b} \psi_j^{a,b} - T_{a,b} \psi_j^{a,b} \frac{dk}{dT} \psi_m \label{dT_dT}
\end{align}

Equations \ref{dT_du} and \ref{dT_dT} above will result in the following terms for LMNS.C:\\
$KTu, KTv, KTw, KTT$

\end{document}
